\documentclass{article}
\usepackage[T1]{fontenc}
\usepackage[polish]{babel}
\usepackage[utf8]{inputenc}
\usepackage{lmodern}
\usepackage{lipsum}
\usepackage{amsmath}
\usepackage{listings}
\usepackage{graphicx} 
\usepackage[margin=1in,left=1.5in,includefoot]{geometry}
\newcommand{\blank}[1]{\hspace*{#1}}
\selectlanguage{polish}
\author{Michal Kolendo}
% header and footer stuff
\usepackage{fancyhdr}
\pagestyle{fancy}
\fancyhead{}
\fancyfoot{}
\fancyfoot[R]{\thepage\\}
\renewcommand{\headrulewidth}{0pt}
\renewcommand{\footrulewidth}{1pt}
%
 
\begin{document}
 
\begin{titlepage}
\begin{center}
\line(1,0){400}\\
[0.25in]
\huge{\bfseries Sprawozdanie}\\
[2mm]
\line(1,0){300}\\
[1.5cm]
\textsc{\LARGE Problem Generowania Stołówki}\\
[0.75cm]
\textsc{\Large Algorytmy i Struktury Danych}\\
[10cm]
\vspace{-9cm}
\textsc{Java}\\
[1cm]
\vspace{9cm}
\end{center}
\begin{flushright}
\textsc{\large Michał Kolendo\\
\small Nr indeksu 286771  \\
\large 26 Styczeń, 2017}
\textsc{\large Bartek Królak\\
\small Nr indeksu 284922  \\
\large 26 Styczeń, 2018}
\end{flushright}
\end{titlepage}
% Front matter stuff
\pagenumbering{arabic}
 
 
 
%this is main body stuff
\setcounter{page}{1}
\section{\underline{Diagram Klas}}
		\begin{figure}[h]
		\hspace*{-5.3cm} 
				\includegraphics[width=1.4\textwidth]{Diagram Klas.png}
				\caption[Diagram Klas programu Time2Dine] {{\sl Diagram Klas programu Time2Dine}}
				\label{Diagram Klas programu Time2Dine}
		\end{figure}
Z powodu wielkości diagramu obejmującego cały projekt wraz z metodami oraz polami, został on zamieszczony w załączniku do projektu pod nazwą \verb|Pełny Diagram UML programu Eggcelent|


\newpage
\section{\underline{Opis zmian w klasach}}

\begin{center}
\vspace{3mm}
\underline{\huge\textbf{{Pakiet Model}}}
\end{center}
\subsection{Chromosome}
\indent Moduł \verb|Chromosome| uległ srednim zmianom.
Zrezygnowaliśmy z przechowywania informacji binarnej na temat rozłożenia mebli w Chromosomie, gdyż została zmieniona metoda mutacji oraz krzyżowania się chromosomów ze sobą. Funkcję generowania chromosomu z \verb|Algorithm| przenieśliśmy również do klasy \verb|Chromosome|.
\subsection{INCheckFile}
\indent Ze zmianą ciała metody w \verb|scanFile(String path)| , musiał zostać zmieniony interfejs. Zmianom uległ prototyp funkcji \verb|scanFile| w którym dodałem warunek rzucania wyjątkiem.
\subsection{LoadFile}
\indent Klasa \verb|LoadFile| nie zaznała poważniejszych zmian. Wprowadziłem do niej zmienną  typu \verb|Graph| ,według założeń z \verb|Specyfikacja Implementacyjna AiSD| , do której wczytuję dane z pliku wejściowego.
\begin{center}
\vspace{3mm}
\underline{\huge\textbf{{Pakiet COUNT}}}
\end{center}
\subsection{Algorithm}
\indent W tej klasie zrezygnowaliśmy z funkcji \verb|generateChromosome(Canteen canteen)| na rzecz klasy \verb|Chromosome|. Argumentami funkcji krzyżującej oraz mutującej Chromosom stała się \verb|ArrayList<Chromosome> chromosomes|.
\subsection{INAlgorithm}
\indent Prototypy funkcji \verb|void mutate(ArrayList<Chromosome> chromosomes)| oraz \verb|chromosome crossBreed(ArrayList<Chromosome> chromosomes)| zmieniły argument jaki przyjmują. W funkcji krzyżującej postanowiliśmy zwracać chromosom powstały z krzyżowania i mutacji.
\begin{center}
\vspace{3mm}
\underline{\huge\textbf{{Nowe Klasy}}}
\subsection{FurnitureEnum}
\subsection{Model}
\subsection{Controller}
\end{center}
\newpage
\section{\textbf{\underline{Opis algorytmu}}}

\indent  Algorytm , który opracowałem w \verb|Specyfikacja Implementacyjna - AiSD| został całkowicie zmieniony. Zrezygnowałem z używania algorytmu Dijkstry przy wyszukiwaniu najkrótszej ścieżki. Problem z którym się mierzyłem udało się rozwiązać znacznie prościej. Z racji braku połączeń pomiędzy fermami oraz pomiędzy sklepami, graf poszukiwań znacznie się upraszczał. Znajdowałem najkrótszą drogą dla każdego sklepu, kierując się najpierw kosztem przewozu jednego jajka, a w drugiej kolejności przepustowością danego połączenia. Przesyłałem nią największą możliwą liczbę jaj z danej fermy do danego sklepu. W przypadku ,gdy sklep był zapełniony, ferma była pusta lub przepustowość drogi wynosiła zero , dany obiekt był usuwany z odpowiadającej mu listy. Dodatkowo , przy usuwaniu sklepu lub fermy, usuwane były również wszystkie połączenia tego obiektu z reszta.\\
\begin{center}
\underline{Komplikacje}
\end{center}
Niestety, nie przeanalizowawszy dogłębnie całej sytuacji, nie zauważyłem możliwości niezapełnienia wszystkich sklepów. Pojawia się ona w przypadku ,gdy wykorzystamy najwygodniejsze ścieżki z \verb|n-1| (gdzie n to liczba hodowli) ferm , a pozostała nie ma połączenia dostatecznie zaopatrzającego pozostały niepełny sklep.
\begin{center}
\underline{Solucja}
\end{center}
Niech \verb|x| oznacza liczbę brakujących jaj w sklepie.\\
Po zdiagnozowaniu problemu,przemyślałem jak należy poprawić sytuację, aby sklepy wszystkie były zapełnione. Tworzyłem liste dróg z fermy w której zostały jajka oraz usuwałem z niej te ,którymi nie prześlę \verb|x| jaj. Z pozostałych dróg wybierałem najtańszą. Sprawdzałem do jakiej fermy prowadzi. Następnie z pustych ferm, analizowałem \verb|"zużyte"|  połączenia w których zostało przesłanych \verb| >= x| jaj.  W przypadku istnienia takiego połączenia, sprawdzałem czy istnieje droga między daną fermą do sklepu z brakiem, którą mamy możliwość przesłać \verb|x| jaj. Jeśli wszystkie warunki zostały spełnione, znależliśmy rozwiązanie naszego problemu. Wiemy z którego sklepu zabrać jajka, aby można było je przesłać do sklepu z brakiem. Dodatkowo uzupełniamy sklep, z którego zabraliśmy połączeniem wybranym na początku.
\subsection{Pliki dodatkowe}
Program wykorzystywał dwa pliki dodatkowe będącymi przykładowymi danymi wejścia w programie:
\begin{itemize}
\item \verb|eggData.txt|;
\item \verb|test.txt|.
\end{itemize}.
\newpage
\section{\underline{Testy programu}}
\indent By wystrzec się błędów zrobiłem kilkanaście testów jednostkowych. Ich wynik znajduje się poniżej:
\subsection{Testy jednostkowe}
\indent Do wykonania testów jednostkowych została wykorzystana biblioteka \verb|Junit4|. Testy jednostkowe zostały stworzone do najważniejszych klas wpływających na sukces wykonania algorytmu. Była to klasa \verb|CheckFile|, klasa \verb|LoadFile|,  klasa \verb|Graph| oraz w małym stopniu klasa \verb|Deliver|.
Zdjęcia z wykonania testów:\\
\vspace{1cm}
\begin{figure}[h]
		\hspace*{2.7cm} 
			\includegraphics[width=0.5\textwidth]{Testy.png}
				\caption[Testy jednostkowe] {{\sl Testy jednostkowe}}
				
		\end{figure}
\newpage

\subsection{Testy całościowe}
\indent Do wykonania testów całościowych użyłem przede wszystkim przykładowych danych zamieszczonych w treści projektu oraz dla danych wymyślonych przeze mnie.\\
 Przykład testu całościowego bazowego przykładu:
\vspace{1cm}
\begin{figure}[h]
\hspace*{-1.5cm} 
\includegraphics[width=1.2\textwidth]{TestCalosciowyAISD.png}
\caption[Test Całościowy] {{\sl Test Całościowy}}
\end{figure}
\section{\underline{Przykładowe użycie programu}}
\indent Zaprezentowane zostaną tu 2 przykładowe użycia programu z konsoli w systemie Windows:
\begin{enumerate}
	\item Pierwsze użycie dla danych bazowych:
Wywyołanie programu:

\verb|java -jar Eggcelent.jar C:\Users\Michal\IdeaProjects\Eggcelent\src\load\eggData.txt|


\begin{figure}[h]
\hspace*{-2.8cm} 
\includegraphics[width=1.3\textwidth]{wynik1.png}
\end{figure}
\newpage
\item Drugie użycie dla innych danych:

Wywołanie programu:\\
\verb|java -jar Eggcelent.jar C:\Users\Michal\IdeaProjects\Eggcelent\src\load\eggData3.txt|

Z powodu ogromnej liczby danych, pokazana zostanie część wyniku:
\begin{figure}[h]
\hspace*{2.4cm} 
\includegraphics[width=0.7\textwidth]{wynik2.png}
\end{figure}

W tym przypadku dla pokazania ogromu transportu jajek, wypisana została liczba jaj.
\end{enumerate}
\section{Kompilacja}
Program jest przeznaczony na dowolony system operacyjny. Ze wzglęgu na to ,iż wirtualna maszyna Java zajmuję się kompilacją programu, a nie system operacyjny na którym program jest, możemy używać Symulatora gdziekolwiek. Wystarczy stworzyć odpowiedni plik \verb |.jar| projektu. Ostatnim krokiem będzie uruchomienie programu z linii poleceń z podaniem argumenty ze ścieżką. W celu bezproblematycznego wyświetlania polskich znaków w konsoli, zaleca się użycie komendy \verb|chcp 65001| ,dzieki której zmieniamy kodowanie otwartej konsoli na \verb|UTF-8|.

\section{Podsumowanie}
Pozytywnie zaskoczony byłem tym, iż diagram UML programu \verb|Eggcelent| zgadzał się z moimi założeniami w \verb|Specyfikacja Implementacyjna AiSD|. Zadanie z którym przyszło mi się zmierzyć nie było łatwe. Analizując przebieg mojej pracy nad tym projektem, zauważyłem wiele rzeczy ,które mogłem zrobić inaczej. 
Jedne z kluczowych jakie zapadną mi w pamięć to:
\begin{itemize}
\item Rozpoczęcie implementacji od modułów ,które wydają się najtrudniejsze, a nie od najprostszych;
\item Dogłębne przeanalizowanie problemu przed przystąpieniem do implementacji (w moim przypadku nie rozważyłem przypadku, iż nie wszystkie sklepy się napełnią);
\item Zwracanie uwagi na kodowanie plików z danymi.
\end{itemize}
Algorytm zaimplementowany przeze mnie nie jest idealny, lecz udostępnia możliwe dobre rozwiązanie problemu. Jestem w pełni świadom ,iż skutecznym byłoby udoskonalenie go o lepszą poprawę w wariancie napełnianiu sklepów.
\end{document}
